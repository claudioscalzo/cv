%-------------------------------------------------------------------------------
%	SECTION TITLE
%-------------------------------------------------------------------------------
\cvsection{Projects}


%-------------------------------------------------------------------------------
%	SUBSECTION TITLE
%-------------------------------------------------------------------------------
% \cvsubsection{English}


%-------------------------------------------------------------------------------
%	CONTENT
%-------------------------------------------------------------------------------
\begin{cventries}

%---------------------------------------------------------
	\cvproj
		{\href{https://github.com/claudioscalzo/coiote-oma-project}{GitHub.com/ClaudioScalzo/CoIoTe-OMA-Project}}
		{Team leader for an optimization project for TIM / SWARM Joint Open Lab}
		{}
		{}
		{
			\begin{cvitems} % Description(s) bullet points
				\item {Solving of a \textit{VRP (Vehicle Routing Problem)} optimization problem proposed by \textit{TIM} and the \textit{SWARM Joint Open Lab}, for an IOT project named \textit{CoIoTe}. Multistart tabu-search approach, written in Java (with the \textit{OpenTS} Java library).}
				\item {Achieved first position in the final ranking. Secured great comprehension of metaheuristics and, overall, improved logical/algorithmic competences and team-working skills.}
			\end{cvitems}
		}

	\cvproj
		{\href{https://github.com/D2KLab/music-chatbot}{GitHub.com/d2kLab/Music-ChatBot}}
		{Virtual Assistant for answering music related questions}
		{}
		{}
		{
			\begin{cvitems} % Description(s) bullet points
				\item {Developing of a virtual assistant (in the chatbot and vocal assistant forms), capable of answering music related questions and providing detailed graphical results. Informations extracted from the \textit{DOREMUS} knowledge base, queried using the \textit{SPARQL} language.}
				\item {Built using \textit{Node.js} code with the \textit{BotKit} framework. Used and trained Google's \textit{Dialogflow} as NLP. Interfaced with the \textit{Facebook Messanger} and \textit{Slack} clients, using the respective APIs.}
			\end{cvitems}
		}
	
	\cvproj
		{\href{https://github.com/claudioscalzo/aml/tree/master/challenge}{GitHub.com/ClaudioScalzo/aml/tree/master/challenge}}
		{House prices \textit{Kaggle} challenge: predicting sales prices with advanced regression techniques}
		{}
		{}
		{
			\begin{cvitems} % Description(s) bullet points
				\item {Solution of one of the most famous Kaggle challenges, developed during the \textit{AML (Algorithmic Machine Learning)} course at \textit{EURECOM}.}
				\item {Top x\% on the global ranking, thanks to smart preprocessing techniques (like \textit{PCA} and \textit{DBSCAN} for outlier removal), and thanks to advanced stacked regressors models.}
				\item {Written in \textit{Python}, using \textit{Pandas DataFrames} for the data structures and \textit{scikit-learn} for the modeling phase.}
			\end{cvitems}
		}
	
	\cvproj
		{\href{https://github.com/claudioscalzo/asi-challenge}{GitHub.com/ClaudioScalzo/asi-challenge}}
		{Challenge on the \texttt{Fashion-MNIST} and \texttt{CIFAR-10} datasets: Naive Bayes and Bayesian Linear Regression}
		{}
		{}
		{
			\begin{cvitems} % Description(s) bullet points
				\item {Solution of the \textit{ASI (Advanced Statistical Inference)} course. Implemented (from scratch) the Naive Bayes Classifier and the Bayesian Linear Regression. Used in the classification tasks of the Fashion-MNIST and CIFAR-10 images datasets.}
				\item {Written in \textit{Python} using \textit{NumPy} and \textit{Pandas}. Achieved extremely satisfying results in terms of accuracy and computational efficiency.}
			\end{cvitems}
		}
%---------------------------------------------------------
\end{cventries}
