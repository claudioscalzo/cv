%-------------------------------------------------------------------------------
%	SECTION TITLE
%-------------------------------------------------------------------------------
\cvsection{Projects}


%-------------------------------------------------------------------------------
%	SUBSECTION TITLE
%-------------------------------------------------------------------------------
% \cvsubsection{English}


%-------------------------------------------------------------------------------
%	CONTENT
%-------------------------------------------------------------------------------
\begin{cventries}

%---------------------------------------------------------
	\cvproj
        {2018~~~|~~~\href{https://github.com/D2KLab/music-chatbot}{github.com/d2klab/music-chatbot}~~~\textbullet~~~\href{https://chatbot.doremus.org}{chatbot.doremus.org}}
        {Virtual Assistant for answering music related questions}
        {}
        {}
        {
            \begin{cvitems} % Description(s) bullet points
                \item {Virtual assistant development (in the chatbot and vocal assistant forms), capable of answering music related questions and providing detailed graphical results. Informations extracted from the \textit{DOREMUS} knowledge base, queried using the SPARQL language.}
                \item {Built with Node.js, using the BotKit framework. Trained Google Dialogflow as NLP. Facebook Messenger, Slack and Google Assistant support.}
                \item {Contributed with two accepted pull requests to the \texttt{botkit-middleware-dialogflow} author, for concurrency and language support.}
            \end{cvitems}
        }

%	\cvproj
%		{2016~~~|~~~\href{https://github.com/claudioscalzo/coiote}{github.com/claudioscalzo/coiote}}
%		{Team leader for an optimization project for TIM / SWARM Joint Open Lab}
%		{}
%		{}
%		{
%			\begin{cvitems} % Description(s) bullet points
%				\item {Solving of a \textit{VRP (Vehicle Routing Problem)} optimization problem proposed by \textit{TIM} and the \textit{SWARM Joint Open Lab}, for an IOT project named \textit{CoIoTe}. Multistart tabu-search approach, written in Java (with the \textit{OpenTS} Java library).}
%				\item {Achieved 1st position in the final ranking. Secured great comprehension of metaheuristics. Improved algorithmic and team-working skills.}
%			\end{cvitems}
%		}
	
%	\cvproj
%		{\href{https://github.com/lomluca/aml}{github.com/lomluca/aml}}
%		{House prices \textit{Kaggle} challenge: predicting sales prices with advanced regression techniques}
%		{}
%		{}
%		{
%			\begin{cvitems} % Description(s) bullet points
%				\item {Solution of the known Kaggle challenge. Achieved top grade on the course ranking thanks to smart preprocessing techniques (like \textit{PCA} and \textit{DBSCAN} for outlier removal), and stacked tree-based and regularized regression models.}
%				\item {Written in \textit{Python}, using \textit{Pandas DataFrames} for the data structures and \textit{scikit-learn} for the modeling phase.}
%			\end{cvitems}
%		}
	
%	\cvproj
%		{\href{https://github.com/claudioscalzo/asi-challenge}{github.com/claudioscalzo/asi-challenge}}
%		{Challenge on the \texttt{Fashion-MNIST} and \texttt{CIFAR-10} datasets: Naive Bayes Classifier and Bayesian Linear Regression}
%		{}
%		{}
%		{
%			\begin{cvitems} % Description(s) bullet points
%				\item {Solution of the \textit{ASI (Advanced Statistical Inference)} course challenge. Implemented (from scratch) the Naive Bayes Classifier and the Bayesian Linear Regression, exploited in the classification tasks of the \texttt{Fashion-MNIST} and \texttt{CIFAR-10} images datasets.}
%				\item {Written in \textit{Python} using \textit{NumPy} and \textit{Pandas}. Achieved extremely satisfying results in terms of accuracy and computational efficiency.}
%			\end{cvitems}
%		}

%	\cvproj
%        {2018~~~|~~~\href{https://github.com/lomluca/aml/tree/master/rec-sys}{github.com/lomluca/aml/tree/master/rec-sys}}
%        {Music Recommendation System with Spark MLlib and with a custom approach}
%        {}
%        {}
%        {
%            \begin{cvitems} % Description(s) bullet points
%                \item {Development of a music recommendation system, starting from a \textit{Last.fm} dataset containing user IDs, artist IDs and the related play counts.}
%                \item {Written in Python, following and comparing two different paths: the Spark MLlib \textit{ALS (Alternating Least Squares)} algorithm and a custom algorithm which, for this task, outperformed the \textit{ALS} performances.}
%            \end{cvitems}
%        }
%---------------------------------------------------------
\end{cventries}
