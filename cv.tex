%!TEX TS-program = lualatex
%!TEX encoding = UTF-8 Unicode

% This CV has been created by:
%   Claudio Scalzo <claudio.scalzo@outlook.com>
%
% using the template:
% 	Awesome CV LaTeX Template for CV/Resume
% 	https://github.com/posquit0/Awesome-CV
% 	Claud D. Park <posquit0.bj@gmail.com> (http://www.posquit0.com)
% 	CC BY-SA 4.0 (https://creativecommons.org/licenses/by-sa/4.0/)


%-------------------------------------------------------------------------------
% CONFIGURATIONS
%-------------------------------------------------------------------------------
% A4 paper size by default, use 'letterpaper' for US letter
\documentclass[11pt, a4paper]{awesome-cv}

% Configure page margins with geometry
\geometry{left=1.4cm, top=.8cm, right=1.4cm, bottom=1.4cm, footskip=.5cm}

% Specify the location of the included fonts
\fontdir[fonts/]

% Color for highlights
% Awesome Colors: awesome-emerald, awesome-skyblue, awesome-red, awesome-pink, awesome-orange
%                 awesome-nephritis, awesome-concrete, awesome-darknight
%\colorlet{awesome}{awesome-red}
\definecolor{awesome}{HTML}{1a3b90} % CLAUDIO

% Colors for text
% Uncomment if you would like to specify your own color
% \definecolor{darktext}{HTML}{414141}
% \definecolor{text}{HTML}{333333}
% \definecolor{graytext}{HTML}{5D5D5D}
% \definecolor{lighttext}{HTML}{999999}
\definecolor{darktext}{HTML}{000000}
\definecolor{text}{HTML}{000000}
\definecolor{graytext}{HTML}{000000}
\definecolor{lighttext}{HTML}{000000}

% Set false if you don't want to highlight section with awesome color
\setbool{acvSectionColorHighlight}{true}

% If you would like to change the social information separator from a pipe (|) to something else
%\renewcommand{\acvHeaderSocialSep}{\quad\textbar\quad}
\renewcommand{\acvHeaderSocialSep}{~~~\textbullet~~~} % CLAUDIO

%-------------------------------------------------------------------------------
%	PERSONAL INFORMATION
%	Comment any of the lines below if they are not required
%-------------------------------------------------------------------------------
% Available options: circle|rectangle,edge/noedge,left/right
\photo[circle,noedge,right]{./images/profile.jpg}
\name{Claudio}{Scalzo}
\position{DevOps Engineer \& Cloud Architect}
%\address{}

\mobile{(+39) 346 245 51 16}
\email{claudio.scalzo@outlook.com}
\github{github.com/claudioscalzo}
\linkedin{linkedin.com/in/claudioscalzo}
% \homepage{website-address}
% \gitlab{gitlab-id}
% \stackoverflow{SO-id}{SO-name}
% \twitter{@twit}
% \skype{skype-id}
% \reddit{reddit-id}
% \extrainfo{extra informations}

\quote{Born in Italy in 1994. Passionate about the whole Cloud Computing and DevOps landscapes.\\Continuous-learner. Strong analytical thinking and enthusiastic approach. Software engineer.\\AWS-certified Solutions Architect and DevOps Engineer. Languages: Italian (native), English (C1 advanced).}


%-------------------------------------------------------------------------------
\begin{document}

% Print the header with above personal informations
% Give optional argument to change alignment(C: center, L: left, R: right)
\makecvheader[L]

% Print the footer with 3 arguments(<left>, <center>, <right>)
% Leave any of these blank if they are not needed
\makecvfooter
  {\href{https://www.linkedin.com/in/claudioscalzo}{Claudio Scalzo}}
  {}
  {\href{https://github.com/claudioscalzo/cv}{GitHub.com/ClaudioScalzo/CV}}


%-------------------------------------------------------------------------------
%	CV/RESUME CONTENT
%	Each section is imported separately, open each file in turn to modify content
%-------------------------------------------------------------------------------
%-------------------------------------------------------------------------------
%	SECTION TITLE
%-------------------------------------------------------------------------------
\cvsection{Experience}


%-------------------------------------------------------------------------------
%	CONTENT
%-------------------------------------------------------------------------------
\begin{cventries}


%---------------------------------------------------------
  \cventry
    {DevOps Engineer \& AWS Cloud Solutions Architect} % Job title
    {Storm Reply} % Organization
    {Turin, Italy} % Location
    {Feb 2019 - now} % Date(s)
    {
        \begin{cvitems} % Description(s) of tasks/responsibilities
\item As a DevOps engineer: implemented CI/CD pipelines, authored and managed infrastructure-as-code declarative templates, deployed enterprise-ready toolchains exploiting proprietary/open-source/AWS-managed tools, implemented serverless backend software and REST API definitions, developed automation software for cloud-based applications. Designed GitOps pipelines, improving critical enterprise flows.
\item As a Solutions Architect: designed cloud-native architectures, architected microservices-based platforms, developed serverless workflows, led cloud migrations and disaster recovery projects. Handled enterprise cloud assets with focus on scalability, reliability, infrastructure and network security, auditing and monitoring, as well as cost optimization aspects.
        \end{cvitems}
    }

  \cventry
    {Software Engineer / Data Engineer intern} % Job title
    {SAP France} % Organization
    {Paris, France} % Location
    {Jul 2018 - Dec 2018} % Date(s)
    {
      \begin{cvitems} % Description(s) of tasks/responsibilities
\item Worked on a Microsoft Azure cloud-based Data Lake platform, implementing a framework to extend the ETL pipeline with a pre-processing and post-processing phase, respectively exploiting object-storage parallel data transfer and machine learning (Spark ML) capabilities.
\item Made use of Spark parallelisation, leveraging distributed computing, the Azure ADLS object storage, and the SAP HANA data warehouse.
\item Managed Jenkins continuous integration and delivery pipelines for test, staging and production environments.
      \end{cvitems}
    }
%---------------------------------------------------------
\end{cventries}

%-------------------------------------------------------------------------------
%	SECTION TITLE
%-------------------------------------------------------------------------------
\cvsection{Education}


%-------------------------------------------------------------------------------
%	CONTENT
%-------------------------------------------------------------------------------
\begin{cventries}

%---------------------------------------------------------
  \cventry
    {Eurecom / Telecom ParisTech} % Institution
    {\textlf{Double Degree in} Data Science and Engineering} % Degree
    {Sophia Antipolis, France} % Location
    {Sep. 2017 - Apr. 2019} % Date(s)
    {
      \begin{cvitems} % Description(s) bullet points
        \item {Double Degree program between Telecom ParisTech and Politecnico di Torino}
      \end{cvitems}
    }
  
  \cventry
    {Politecnico di Torino}
    {\textlf{Master of Science in} Software Engineering}
    {Turin, Italy}
    {Sep. 2016 - Apr. 2019}
    {
      \begin{cvitems}
        \item {with the highest grade, \textsc{110/110} with honors}
      \end{cvitems}
    }
  
  \cventry
    {Politecnico di Torino}
    {\textlf{Bachelor's Degree in} Software Engineering}
    {Turin, Italy}
    {Sep. 2013 - Jul. 2016}
    {
      \begin{cvitems}
        \item {with the highest grade, \textsc{110/110}}
      \end{cvitems}
    }

%---------------------------------------------------------
\end{cventries}

%-------------------------------------------------------------------------------
%    SECTION TITLE
%-------------------------------------------------------------------------------
\cvsection{Skills}


%-------------------------------------------------------------------------------
%    CONTENT
%-------------------------------------------------------------------------------
\begin{cvskills}

%---------------------------------------------------------
  \cvskill
    {Languages} % Category
    {Python~~~\textbullet~~~C~~~\textbullet~~~Java~~~\textbullet~~~SQL (Oracle PL/SQL, HiveQL)~~~\textbullet~~~SPARQL~~~\textbullet~~~MATLAB~~~\textbullet~~~Bash / AWK / Sed} % Skills

%---------------------------------------------------------
  \cvskill
    {Big Data} % Category
    {Apache Spark~~~\textbullet~~~Apache Hive~~~\textbullet~~~MapReduce \& HDFS~~~\textbullet~~~NoSQL Architectures} % Skills
    
%---------------------------------------------------------
\cvskill
    {Cloud Computing} % Category
    {Microsoft Azure (HDInsight, ADLS)~~~\textbullet~~~Apache Ambari} % Skills

%---------------------------------------------------------
  \cvskill
    {Machine Learning} % Category
    {Spark ML~~~\textbullet~~~scikit-learn~~~\textbullet~~~Keras~~~\textbullet~~~Deep Learning (NNs, CNNs, RNNs)~~~\textbullet~~~Probabilistic Machine Learning~~~\textbullet~~~NLP} % Skills

%---------------------------------------------------------
  \cvskill
    {VCS, CI, CD \& Others} % Category
    {Git~~~\textbullet~~~Jenkins CI/CD~~~\textbullet~~~Jupyter / Zeppelin Notebooks~~~\textbullet~~~Dialogflow (+ BotKit)} % Skills

%---------------------------------------------------------
\end{cvskills}

%-------------------------------------------------------------------------------
%	SECTION TITLE
%-------------------------------------------------------------------------------
\cvsection{Certifications}


%-------------------------------------------------------------------------------
%	SUBSECTION TITLE
%-------------------------------------------------------------------------------
% \cvsubsection{English}


%-------------------------------------------------------------------------------
%	CONTENT
%-------------------------------------------------------------------------------
\begin{cventries}

%---------------------------------------------------------
	\cvproj
        {AWS Certifications}
        {}
        {}
        {}
        {
            \begin{cvitems}
                \item \textbf{AWS Certified Solutions Architect - Associate Level}~~~\textbullet~~~\href{https://github.com/claudioscalzo/cv/raw/master/documents/aws\_csa\_associate.pdf}{github.com/claudioscalzo/cv/raw/master/documents/aws\_csa\_associate.pdf}
                \item \textbf{AWS Certified Cloud Practitioner}~~~\textbullet~~~\href{https://github.com/claudioscalzo/cv/raw/master/documents/aws\_ccp.pdf}{github.com/claudioscalzo/cv/raw/master/documents/aws\_ccp.pdf}
            \end{cvitems}
        }
%---------------------------------------------------------
\end{cventries}

%%-------------------------------------------------------------------------------
%	SECTION TITLE
%-------------------------------------------------------------------------------
\cvsection{Projects}


%-------------------------------------------------------------------------------
%	SUBSECTION TITLE
%-------------------------------------------------------------------------------
% \cvsubsection{English}


%-------------------------------------------------------------------------------
%	CONTENT
%-------------------------------------------------------------------------------
\begin{cventries}

%---------------------------------------------------------
	\cvproj
		{\href{https://github.com/claudioscalzo/coiote}{github.com/claudioscalzo/coiote}}
		{Team leader for an optimization project for TIM / SWARM Joint Open Lab}
		{}
		{}
		{
			\begin{cvitems} % Description(s) bullet points
				\item {Solving of a \textit{VRP (Vehicle Routing Problem)} optimization problem proposed by \textit{TIM} and the \textit{SWARM Joint Open Lab}, for an IOT project named \textit{CoIoTe}. Multistart tabu-search approach, written in Java (with the \textit{OpenTS} Java library).}
				\item {Achieved 1st position in the final ranking. Secured great comprehension of metaheuristics. Improved algorithmic and team-working skills.}
			\end{cvitems}
		}

	\cvproj
		{\href{https://github.com/D2KLab/music-chatbot}{github.com/d2klab/music-chatbot}~~~\textbullet~~~\href{https://chatbot.doremus.org}{chatbot.doremus.org}}
		{Virtual Assistant for answering music related questions}
		{}
		{}
		{
			\begin{cvitems} % Description(s) bullet points
				\item {Developing of a virtual assistant (in the chatbot and vocal assistant forms), capable of answering music related questions and providing detailed graphical results. Informations extracted from the \textit{DOREMUS} knowledge base, queried using the \textit{SPARQL} language.}
				\item {Built using \textit{Node.js} code with the \textit{BotKit} framework. Used and trained Google's \textit{Dialogflow} as NLP. Interfaced with the \textit{Facebook Messenger}, \textit{Slack} and \textit{Google Assistant} clients, using the respective APIs.}
				\item {Did two pull requests (accepted and merged) to the \texttt{botkit-middleware-dialogflow} author, for concurrency and language support.}
			\end{cvitems}
		}
	
	\cvproj
		{\href{https://github.com/lomluca/aml}{github.com/lomluca/aml}}
		{House prices \textit{Kaggle} challenge: predicting sales prices with advanced regression techniques}
		{}
		{}
		{
			\begin{cvitems} % Description(s) bullet points
				\item {Solution of one of the most famous Kaggle challenges, developed during the \textit{AML (Algorithmic Machine Learning)} course at \textit{EURECOM}.}
				\item {Top grade on the course ranking, thanks to smart preprocessing techniques (like PCA and DBSCAN for outlier removal), and stacked models.}
				\item {Written in \textit{Python}, using \textit{Pandas DataFrames} for the data structures and \textit{scikit-learn} for the modeling phase.}
			\end{cvitems}
		}
	
	\cvproj
		{\href{https://github.com/claudioscalzo/asi-challenge}{github.com/claudioscalzo/asi-challenge}}
		{Challenge on the \texttt{Fashion-MNIST} and \texttt{CIFAR-10} datasets: Naive Bayes Classifier and Bayesian Linear Regression}
		{}
		{}
		{
			\begin{cvitems} % Description(s) bullet points
				\item {Solution of the \textit{ASI (Advanced Statistical Inference)} course. Implemented (from scratch) the Naive Bayes Classifier and the Bayesian Linear Regression. Used in the classification tasks of the Fashion-MNIST and CIFAR-10 images datasets.}
				\item {Written in \textit{Python} using \textit{NumPy} and \textit{Pandas}. Achieved extremely satisfying results in terms of accuracy and computational efficiency.}
			\end{cvitems}
		}
%---------------------------------------------------------
\end{cventries}

%-------------------------------------------------------------------------------
%	SECTION TITLE
%-------------------------------------------------------------------------------
\cvsection{Accomplishments}


%-------------------------------------------------------------------------------
%	SUBSECTION TITLE
%-------------------------------------------------------------------------------
% \cvsubsection{English}


%-------------------------------------------------------------------------------
%	CONTENT
%-------------------------------------------------------------------------------
\begin{cvskills}
    
\cvskill
    {AWS Migration GameDay Milan 2019}{1st place}
\cvskill{Reply Xchange 2020}{Public Speech on ``GitOps \& Continuous Everything", \textasciitilde500 attendees}
        
\end{cvskills}



%-------------------------------------------------------------------------------
\end{document}
